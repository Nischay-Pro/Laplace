% CREATED BY DAVID FRISK, 2015
\chapter{Directions to use the code}
\section{General instructions}
The following points directs the user on how to compile and use the code
\\

\begin{itemize}
  \item Clone the code from Github.
  \item In the terminal, enter 'make laplace'.
  \item once the executable is generated type './laplace' to run the code.
  \item The code will output the number of iterations and final error.
  \item The user can run the code with a different grid size.
  \item The code can be run in parallel(for example, 2 processors) using\\ 'mpiexec -n 2 ./laplace'. 
\end{itemize}

\section{General PETSc guidelines}

\begin{itemize}
  \item The common mistakes arise in not initializing PETSc, missing \\Fortran spacing to name a few.
  \item Always refer the manual before implementing your own code.
  \item Use the PETSc viewing options to understand the numbering \\used and to check for errors.
  \item It is recommended to use Valgrind to check for memory leaks in \\case of any errors.
  \item To understand the fundamental concepts of PETSc and its\\ implementation, go through the tutorials.
  
  
\end{itemize}

