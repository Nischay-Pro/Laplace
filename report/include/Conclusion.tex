% CREATED BY DAVID FRISK, 2015
\chapter{Conclusion}
\section{Results}
\begin{comment}


From figures (\ref{fig:51}-\ref{fig:53}) it is clear that the heat flow dynamics in numerical cases follow that of the analytical case. This can be further concurred through the line plots in figures (\ref{fig:56}-\ref{fig:58}). From figure \ref{fig:56} it can be concluded that both CD and PDDS perform similarly when subjected to the clean grid during the simulations. Also, it can be seen that from the figure (\ref{fig:57}), PDDS is closer to the analytical solution in most of the regions when compared to CD. Further, from the figure (\ref{fig:58}) it can be seen that only PDDS produces good results while CD fails in this case. Therefore, from the obtained results it can be concluded that PDDS performs better than CD when the grid consists of unfavorable cells.
\end{comment}
\section{Further Work}
\begin{comment}
With the results obtained from PDDS being on par with the analytical solution, there is an availability of a working model for this new scheme. This model will be used as a base for implementing this scheme in the in-house CFD code. With the understanding of the behavior of the numerical scheme in capturing the gradients in a domain with unfavorable cells, this scheme will be further tested for several cases at larger scales which involves more variables. This scheme will be further tested with an engineering problem (for example, turbine blade optimization). Finally, a test for robustness and accuracy of the scheme to solve diffusive fluxes for studying the effect of the mesh quality (skewness, stretching, aspect ratio, etc.) will be conducted.
\end{comment}
